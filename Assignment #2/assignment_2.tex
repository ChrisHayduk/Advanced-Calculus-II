\documentclass[12pt]{article}
 
\usepackage[margin=1in]{geometry}
\usepackage{amsmath,amsthm,amssymb, mathtools}
\usepackage[T1]{fontenc}
\usepackage{lmodern}
\usepackage{fixltx2e}
\usepackage[shortlabels]{enumitem}
\usepackage{mathrsfs}
 
\newcommand{\N}{\mathbb{N}}
\newcommand{\R}{\mathbb{R}}
\newcommand{\Z}{\mathbb{Z}}
\newcommand{\Q}{\mathbb{Q}}
 
\newenvironment{theorem}[2][Theorem]{\begin{trivlist}
\item[\hskip \labelsep {\bfseries #1}\hskip \labelsep {\bfseries #2.}]}{\end{trivlist}}
\newenvironment{lemma}[2][Lemma]{\begin{trivlist}
\item[\hskip \labelsep {\bfseries #1}\hskip \labelsep {\bfseries #2.}]}{\end{trivlist}}
\newenvironment{exercise}[2][Exercise]{\begin{trivlist}
\item[\hskip \labelsep {\bfseries #1}\hskip \labelsep {\bfseries #2.}]}{\end{trivlist}}
\newenvironment{problem}[2][Problem]{\begin{trivlist}
\item[\hskip \labelsep {\bfseries #1}\hskip \labelsep {\bfseries #2.}]}{\end{trivlist}}
\newenvironment{question}[2][Question]{\begin{trivlist}
\item[\hskip \labelsep {\bfseries #1}\hskip \labelsep {\bfseries #2.}]}{\end{trivlist}}
\newenvironment{corollary}[2][Corollary]{\begin{trivlist}
\item[\hskip \labelsep {\bfseries #1}\hskip \labelsep {\bfseries #2.}]}{\end{trivlist}}
\newcommand{\textfrac}[2]{\dfrac{\text{#1}}{\text{#2}}}

\begin{document}

\title{Advanced Calculus II: Assignment 2}

\author{Chris Hayduk}
\date{\today}

\maketitle

\begin{problem}{1}
\end{problem}

Let $x, y \in A$. Suppose $x \in C_y$ and $C_x \not\subset C_y$. That is, there are points in $C_x$ which are not connected to $y$.\\

Let $C = C_x \cup C_y$. $C$ is clearly disconnected by the above reasoning. Then there exist open sets $B, D$ such that $C \cap B$, $C \cap D$ are disjoint, non-empty, and have union $C$.\\

Since $C \cap B$, $C \cap D$ are disjoint, $x \in B$ or $x \in D$. Assume $x \in B$ without loss of generality.\\

We know $x \in C_y$ by assumption. Since $C_y$ is connected, it must all be contained in $B$, otherwise it would be split between $C_y \cap B$, $C_y \cap D$ where they are both non-empty, disjoint, and where the union is $C_y$, a contradiction.\\

In addition, we know $C \cap D$ is non-empty. Since $C_y \subset B$, then elements of $C_x$ must be in $D$. However, $x \in B$.\\

Hence, if we take $C_x \cap B$ and $C_x \cap D$, we have two non-empty, disjoint sets with a union equal to $C_x$. However, $C_x$ is connected by definition, so this is a contradiction.\\

Thus, we have that $C_x \subset C_y$.\\

If we swap $x$ with $y$ in the above proof, we get that $C_y \subset C_x$. Hence $C_x = C_y$ if $C_x \cap C_y \neq \emptyset$.
\begin{problem}{2}
\end{problem}

Let $A = \mathbb{I}$, the set of irrational numbers.\\

Since $\mathbb{Q}$ is countable, $\mathbb{R}$ is uncountable, and $\mathbb{R} = \mathbb{Q} \cup \mathbb{I}$, we have that $\mathbb{I}$ must be uncountable. If it were not, then $\mathbb{R}$ would be the union of two countable sets and would hence be countable as well, a contradiction.\\

Now observe that $\mathbb{Q}$ is dense in $\mathbb{R}$. That is, for every $x_1, x_2 \in \mathbb{R}$ with $x_1 < x_2$, $\exists y \in \mathbb{Q}$ such that $x_1 < y < x_2$. Since $\mathbb{I} \subset \mathbb{R}$, we see that $\mathbb{Q}$ divides $\mathbb{I}$ in such a manner.\\

Lastly, from Theorem 12.8 in the text, we know that a subset of $\mathbb{R}$ is connected if and only if it is an interval.\\

Now let $x \in \mathbb{I}$ and let $\epsilon > 0$ be arbitrarily small. Take the interval $[x - \epsilon, x + \epsilon]$. Since there exists a rational number between any two real numbers, and we have that $x, x + \epsilon \in \mathbb{R}$, we have that $\exists y \in \mathbb{Q}$ such that $y \in [x, x + \epsilon]$.\\

Hence, $[x - \epsilon, x + \epsilon] \not\subset \mathbb{I}$. This holds for every $\epsilon > 0$, so there are no interval subsets of $\mathbb{I}$. Since these are the only connected subsets of $\mathbb{R}$, $\mathbb{I}$ is totally disconnected.

\begin{problem}{3}
\end{problem}

Let $C \subset \mathbb{R}^p$ be open and suppose $C$ is connected.\\

Let $x \in C$. Let $U$ be the set of points that can be connected to $x$ by a path in $C$.\\

Let $V = C \setminus U$. That is, $V$ is the set of points in $C$ which cannot be connected to $x$ via some path. So we have that $U \cap V = \emptyset$ and $U \cup V = C$.\\

We will show that U and V are both open.\\

Let $y \in U$ and let $f$ be a path connecting $x$ and $y$. Since $C$ open, $\exists B_r(y) \subset C$ with $r > 0$.\\

For every $z \in B_r(y)$, there is a path $f$ connecting $y$ and $z$.\\

Thus, $B_r(y) \subset U$. Since this holds for each $y \in U$, $U$ is open. A similar argument holds for $V$ showing that it is open.\\

Since $C$ is connected, either $U$ or $V$ must then be the empty set. Assume $V = \emptyset$ without loss of generality. Since $C = U \cup V$ and $V = \emptyset$, we have that $C = U$.\\

Since $U$ is path connected, we then have that $C$ is path connected as required.\\

Now suppose $C$ is open and path connected. Also suppose that it is not connected for contradiction.\\

Then $C = U \cup V$ where $U \cap V = \emptyset$, $U, V$ open, and $(U \cap C) \cup (V \cap C) = C$.\\

Let $x \in U \cap C$ and $y \in V \cap C$. Since $C$ is path connected, $\exists$ a continuous function $f: [0, 1] \to C$ with $f(0) = x$ and $f(1) = y$.\\

Then we have $[0, 1] = f^{-1}(U) \cup f^{-1}(V)$, which would be a nontrivial splitting of $[0, 1]$ by the continuity of $f$. This implies that $[0, 1]$ is disconnected, a contradiction.\\

Hence $C$ open and path connected implies that $C$ is connected.

\begin{problem}{4}
\end{problem}

\begin{problem}{5}
\end{problem}

Let $T$ be the topologist's sine curve. That is,
\begin{align*}
T = \left\{  \left( x, \sin \tfrac{1}{x}  \right ) :  x \in (0,1] \right\} \cup \{(0,0)\}
\end{align*}

$T$ is not path connected because it includes the point $(0, 0)$. However, the function $\left( x, \sin \tfrac{1}{x}  \right )$ cannot be extended to include this point such that it forms a path.\\

Now observe that $A = \left\{  \left( x, \sin \tfrac{1}{x}  \right ) :  x \in (0,1] \right\}$ is path connected and hence connected.\\

In addition, observe that the topologist's sine curve is the closure of $A$.\\

Thus, $A \subset T \subset \overline{A}$.\\

As a result, we have that $T$ is connected.

\end{document}