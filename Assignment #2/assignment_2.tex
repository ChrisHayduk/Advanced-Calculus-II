\documentclass[12pt]{article}
 
\usepackage[margin=1in]{geometry}
\usepackage{amsmath,amsthm,amssymb, mathtools}
\usepackage[T1]{fontenc}
\usepackage{lmodern}
\usepackage{fixltx2e}
\usepackage[shortlabels]{enumitem}
\usepackage{mathrsfs}
 
\newcommand{\N}{\mathbb{N}}
\newcommand{\R}{\mathbb{R}}
\newcommand{\Z}{\mathbb{Z}}
\newcommand{\Q}{\mathbb{Q}}
 
\newenvironment{theorem}[2][Theorem]{\begin{trivlist}
\item[\hskip \labelsep {\bfseries #1}\hskip \labelsep {\bfseries #2.}]}{\end{trivlist}}
\newenvironment{lemma}[2][Lemma]{\begin{trivlist}
\item[\hskip \labelsep {\bfseries #1}\hskip \labelsep {\bfseries #2.}]}{\end{trivlist}}
\newenvironment{exercise}[2][Exercise]{\begin{trivlist}
\item[\hskip \labelsep {\bfseries #1}\hskip \labelsep {\bfseries #2.}]}{\end{trivlist}}
\newenvironment{problem}[2][Problem]{\begin{trivlist}
\item[\hskip \labelsep {\bfseries #1}\hskip \labelsep {\bfseries #2.}]}{\end{trivlist}}
\newenvironment{question}[2][Question]{\begin{trivlist}
\item[\hskip \labelsep {\bfseries #1}\hskip \labelsep {\bfseries #2.}]}{\end{trivlist}}
\newenvironment{corollary}[2][Corollary]{\begin{trivlist}
\item[\hskip \labelsep {\bfseries #1}\hskip \labelsep {\bfseries #2.}]}{\end{trivlist}}
\newcommand{\textfrac}[2]{\dfrac{\text{#1}}{\text{#2}}}

\begin{document}

\title{Advanced Calculus II: Assignment 2}

\author{Chris Hayduk}
\date{\today}

\maketitle

\begin{problem}{1}
\end{problem}

\begin{problem}{2}
\end{problem}

Let $A = \mathbb{I}$, the set of irrational numbers.\\

Since $\mathbb{Q}$ is countable, $\mathbb{R}$ is uncountable, and $\mathbb{R} = \mathbb{Q} \cup \mathbb{I}$, we have that $\mathbb{I}$ must be uncountable. If it were not, then $\mathbb{R}$ would be the union of two countable sets and would hence be countable as well, a contradiction.\\

Now observe that $\mathbb{Q}$ is dense in $\mathbb{R}$. That is, for every $x_1, x_2 \in \mathbb{R}$ with $x_1 < x_2$, $\exists y \in \mathbb{Q}$ such that $x_1 < y < x_2$. Since $\mathbb{I} \subset \mathbb{R}$, we see that $\mathbb{Q}$ divides $\mathbb{I}$ in such a manner.\\

\begin{problem}{3}
\end{problem}

Let $C \subset \mathbb{R}^p$ be open and suppose $C$ is connected.

\end{document}