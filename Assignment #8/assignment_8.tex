\documentclass[12pt]{article}
 
\usepackage[margin=1in]{geometry}
\usepackage{amsmath,amsthm,amssymb, mathtools}
\usepackage[T1]{fontenc}
\usepackage{lmodern}
\usepackage{fixltx2e}
\usepackage[shortlabels]{enumitem}
\usepackage{mathrsfs}
 
\newcommand{\N}{\mathbb{N}}
\newcommand{\R}{\mathbb{R}}
\newcommand{\Z}{\mathbb{Z}}
\newcommand{\Q}{\mathbb{Q}}
 
\newenvironment{theorem}[2][Theorem]{\begin{trivlist}
\item[\hskip \labelsep {\bfseries #1}\hskip \labelsep {\bfseries #2.}]}{\end{trivlist}}
\newenvironment{lemma}[2][Lemma]{\begin{trivlist}
\item[\hskip \labelsep {\bfseries #1}\hskip \labelsep {\bfseries #2.}]}{\end{trivlist}}
\newenvironment{exercise}[2][Exercise]{\begin{trivlist}
\item[\hskip \labelsep {\bfseries #1}\hskip \labelsep {\bfseries #2.}]}{\end{trivlist}}
\newenvironment{problem}[2][Problem]{\begin{trivlist}
\item[\hskip \labelsep {\bfseries #1}\hskip \labelsep {\bfseries #2.}]}{\end{trivlist}}
\newenvironment{question}[2][Question]{\begin{trivlist}
\item[\hskip \labelsep {\bfseries #1}\hskip \labelsep {\bfseries #2.}]}{\end{trivlist}}
\newenvironment{corollary}[2][Corollary]{\begin{trivlist}
\item[\hskip \labelsep {\bfseries #1}\hskip \labelsep {\bfseries #2.}]}{\end{trivlist}}
\newcommand{\textfrac}[2]{\dfrac{\text{#1}}{\text{#2}}}

\begin{document}

\title{Advanced Calculus II: Assignment 8}

\author{Chris Hayduk}
\date{\today}

\maketitle

\begin{problem}{1}
\end{problem}

Let $F(x, y) = x^2 + y^2 - 1$. Consider the equation $F(x, y) = x^2 + y^2 - 1 = 0$.\\

In order to prove that the above equation can be solved for small values of $x$ by a positive function $y = y(x)$, we need to show that $F$ belongs to class $C^1(\mathbb{R}^2)$, that $F(a, b) = 0$, and that the linear map defined by $L_2(v) = DF(a, b)(0, v)$, $v \in \mathbb{R}$ is a bijection of $\mathbb{R}$ onto $\mathbb{R}$ (by Implicit Function Theorem).\\

First let $(a, b) = (0, 1)$. Observe that $F(a, b) = F(0, 1) = 0^2 + 1^2 - 1 = 0$ as required.\\

Now by Theorem 41.2, in order to show that $F$ belongs to class $C^1(\mathbb{R}^2)$ (ie. the derivative exists and is continuous), it suffices to show that both partial derivatives are continuous on $\mathbb{R}^2$. Hence, we have,
\begin{align*}
D_x F &= 2x\\
D_y F &= 2y
\end{align*}

It is clear that both partial derivatives are continuous on all of $\mathbb{R}^2$ since both are comprised of a multiplication of two trivially continuous functions.\\

Finally, we need to show that the linear map defined by $L_2(v) = DF(0, 1)(0, v)$, $v \in \mathbb{R}$ is a bijection of $\mathbb{R}$ onto $\mathbb{R}$.\\

Note from Corollary 39.7 in the text that,
\begin{align*}
DF(0, 1)(0, v) &= (0)D_xF(0,1) + (v)D_yF(0, 1)\\
&= v (2 \cdot 1)\\
&= 2v
\end{align*}

It is clear that this function is defined on all of $\mathbb{R}$.\\

Now suppose that $\exists v_1, v_2 \in \mathbb{R}$ such that $L_2(v_1) = L_2(v_2)$. Then we have that,
\begin{align*}
&2v_1 = 2v_2\\
\implies &v_1 = v_2
\end{align*}

Hence $L_2$ is injective.\\

Now let $u \in \mathbb{R}$. Observe that, since $\mathbb{R}$ is closed under multiplication and $\frac{1}{2} \in \mathbb{R}$, we have that $\frac{u}{2} \in \mathbb{R}$. Hence, for any $u \in \mathbb{R}, \exists \frac{u}{2}$ such that,
\begin{align*}
L_2\left(\frac{u}{2}\right) &= 2\left(\frac{u}{2}\right) = u
\end{align*}

Thus, $L_2$ is surjective.\\

Since $L_2$ is both injective and surjective, it is a bijective function from $\mathbb{R} \to \mathbb{R}$.\\

Since $F(x, y)$ satisfies all of the above properites, we can apply the Implicit Function Theorem. Hence, there exists an open neighborhood $W$ of $0 \in \mathbb{R}$ and a unique function $y(x): W \to \mathbb{R}$ belonging to class $C^1(W)$ such that $y = y(x)$ and,
\begin{align*}
F(x, y(x)) = 0 \;\; \forall x \in W
\end{align*}

Now we need to show that $y'(x) = \frac{-x}{y}$.\\

Applying Corollary 41.10 from the text yields,
\begin{align*}
Dy(x) &= -[D_{(2)} F(x, y(x))]^{-1} \circ [D_{(1)} F(x, y(x)]\\
&= -\frac{1}{2y} \circ 2x\\
&= -\frac{x}{y}
\end{align*}

as required.

\begin{problem}{2}
\end{problem}

As above, we need to show  we need that $F$ belongs to class $C^1(\mathbb{R}^5)$, that $F(a, b) = 0$, and that the linear map defined by $L_2(v) = DF(a, b)(0, v)$, $v \in \mathbb{R}^2$ is a bijection of $\mathbb{R}^2$ onto $\mathbb{R}^2$ (by Implicit Function Theorem).\\

Observe that $(a, b) = (2, 1, 0, -1, 0)$ and we are given that $F(2, 1, 0, -1, 0) = (0, 0)$. Hence this condition is already satisfied.\\

Next, we need to show that $F$ belongs to class $C^1$. First we will define the derivative of $F$,
\begin{align*}
DF(u, v, w, x, y) &= \begin{bmatrix}
y & v & 1 & v + 2x & u\\
vw & uw & uv & 1 & 1
\end{bmatrix}
\end{align*}

Recall that each partial derivative of $F$ is an entry in the above matrix and that the derivative of $F$ is continuous iff every partial derivative is continuous.\\

From the above, it is clear that all of the partial derivatives are continuous on all of $\mathbb{R}^5$ as they are linear combinations of continuous functions. Hence, $DF(u, v, w, x, y)$ is continuous on all of $\mathbb{R}^5$ and $F$ belongs to class $C^1(\mathbb{R}^5)$\\

Lastly, we need to show that the linear map defined by $L_2(v) = DF(a, b)(0, v)$, $v \in \mathbb{R}^2$ is a bijection of $\mathbb{R}^2$ onto $\mathbb{R}^2$ where $a = (2, 1, 0)$ and $b = (-1, 0)$.\\

Hence, we have,
\begin{align*}
L_2(v) &= \begin{bmatrix}
0 & 1 & 1 & -1 & 2\\
0 & 0 & 2 & 1 & 1
\end{bmatrix} \cdot \begin{bmatrix}
0 \\ 0 \\ 0 \\ v_1 \\ v_2
\end{bmatrix}\\
&= \begin{bmatrix}
-v_1 + 2v_2 \\
v_1 + v_2
\end{bmatrix}
\end{align*}

Now suppose $L_2(v) = L_2(w)$ for $v, w \in \mathbb{R}^2$ with $v \neq w$. Then we have,
\begin{align*}
-v_1 + 2v_2 = -w_1 + 2w_2
\end{align*}

and 
\begin{align*}
v_1 + v_2 = w_1 + w_2
\end{align*}

Subtracting $2$ times the second equation from the first equation yields,
\begin{align*}
v_1 = w_1
\end{align*}

Plugging this identity in yields,
\begin{align*}
v_2 = w_2
\end{align*}

Hence $v = w$ and $L_2$ is injective.\\

Now suppose $w \in \mathbb{R}^2$. We will now consider the equation,
\begin{align*}
L_2(v) = w
\end{align*}

for some $v \in \mathbb{R}^2$, which is equivalent to,
\begin{align*}
\begin{bmatrix}
-v_1 + 2v_2 \\
v_1 + v_2
\end{bmatrix} &= \begin{bmatrix} w_1 \\ w_2 \end{bmatrix}
\end{align*}

This system of equations yields $v_1 = \frac{1}{3} (2w_2 - w_1)$ and $v_2 = \frac{w_1 + w_2}{3}$. Note that $w_1, w_2 \in \mathbb{R}$ and $\mathbb{R}$ is closed under addition and multiplication. Hence, $v_1, v_2 \in \mathbb{R}$ and, by extension, $v \in \mathbb{R}^2$.\\

Thus, for any $w \in \mathbb{R}^2$, $\exists v \in \mathbb{R}^2$ such that $L_2(v) = w$. As a result, $L_2$ is a bijection from $\mathbb{R}^2 \to \mathbb{R}^2$.\\

Thus, $F$ satisfies all of the necessary properties. As a result, we can apply the Implicit Function Theorem. Hence, there exists an open neighborhood $W$ of $(2, 1, 0) \in \mathbb{R}^3$ and a unique function $\varphi(x): W \to \mathbb{R}^2$ belonging to class $C^1(W)$ such that $(x, y)= \varphi(u, v, w)$ and,
\begin{align*}
F(u, v, w, \varphi(u, v, w)) = 0 \;\; \forall (u, v, w) \in W
\end{align*}

Now we need to compute $D\varphi (2, 1, 0)$.\\

We will start by using Corollary 41.10 from the text,
\begin{align*}
D\varphi(u, v, w) &= -[D_{(2)}F(u, v, w, \varphi(u, v, w)]^{-1} \circ [D_{(1)}F(u, v, w, \varphi(u, v, w)]\\
&= \frac{1}{v+2x - u} \begin{bmatrix}
1 & -u\\
-1 & v+2x
\end{bmatrix} \circ \begin{bmatrix}
y & v & 1\\
vw & uw & uv
\end{bmatrix}\\
&= \frac{1}{v+2x - u} \begin{bmatrix}
y - uvw & v - u^2w & 1 - u^2v\\
-y + v^2w + 2xvw & -v + vuw + 2xuw & -1 + uv^2 + 2xuv
\end{bmatrix}
\end{align*}

Hence, we have,
\begin{align*}
D\varphi(2,1,0) &= \frac{1}{1 +2x - 2} \begin{bmatrix}
y - (2)(1)(0) & 1 - 2^2(0) & 1 - 2^2(1)\\
-y + 1^2(0) + 2x(1)(0) & -1 + (1)(2)(0) + 2x(2)(0) & -1 + 2(1^2) + 2x(2)(1)
\end{bmatrix}\\
&= \frac{1}{2x - 1} \begin{bmatrix}
y & 1 & -3 \\
-y & -1 & 4x + 1
\end{bmatrix}
\end{align*}

\begin{problem}{3}
\end{problem}

Let $D \subset \mathbb{R}^2$ be open and $f \in C^1(D, \mathbb{R})$. Let $(x_0, y_0) \in D$. If $Df(x_0, y_0) = 0$ and if $\exists \delta > 0$ such that for every $(x,y)$ with $||(x, y) - (x_0, y_0)|| < \delta$, we have $Df(x, y)(u, v) > 0 \; \; \forall (u, v) \in \mathbb{R}^2$, then $f$ has a local maximum at $(x_0, y_0)$ \\

First, observe that $Df(x_0, y_0) = 0$, and hence $f$ has a critical point at $(x_0, y_0)$. The set of points of relative extrema of $f$ is a subset of the set of critical points of $f$ (as stated on p. 398 in the text), so this is a necessary condition for $f$ to have a relative maximum at $(x_0, y_0)$.\\

Now observe that $Df$ exists for all $x \in D$ and is a continuous mapping of $D$ into $\mathscr{L}(\mathbb{R}^2, \mathbb{R})$ since $f \in C^1(D, \mathbb{R})$.

\begin{problem}{4}
\end{problem}

In order to apply the Implicit Function theorem, we use the following equivalent formulation for this system of equations:
\begin{align*}
F(x, y, u, v) &= (x^2 + y^2 - u^2 - v^2, x^2 + 2y^2 + 3u^2 + 4v^2 - 1) = (0, 0)
\end{align*}

Note that $F(\frac{1}{\sqrt{10}}, \frac{1}{\sqrt{10}}, \frac{1}{\sqrt{10}}, \frac{1}{\sqrt{10}}) = (0, 0)$. We will use this point as our $(a, b)$ in the Implicit Function Theorem.\\

In addition, note that $\Omega = \mathbb{R}^4$ in this case. Hence, $\Omega$ is open as required.\\

We now need to show that $F$ belongs to class $C^1(\mathbb{R}^4)$. Recall that $F$ belongs to class $C^1(\mathbb{R}^4)$ iff all of the first partial derivatives of $F$ are continuous on $\mathbb{R}^4$. We will verify that now,
\begin{align*}
DF(x, y, u, v) &= \begin{bmatrix}
2x & 2y & -2u & -2v\\
2x & 4y & 6u & 8v
\end{bmatrix}
\end{align*}

Again, clearly each partial derivative is continuous on $\mathbb{R}^4$ as they are just degree 1 polynomials. Hence, $F$ belongs to class $C^(\mathbb{R}^4)$.\\

Now we need to show that the linear map defined by $L_2(v) = DF(\frac{1}{\sqrt{10}}, \frac{1}{\sqrt{10}}, \frac{1}{\sqrt{10}}, \frac{1}{\sqrt{10}})(0, v) \;\; v \in \mathbb{R}^2$ is a bijection of $\mathbb{R}^2$ onto $\mathbb{R}^2$.\\

We will start by defining this linear map,
\begin{align*}
L_2(v) &= DF\left(\frac{1}{\sqrt{10}}, \frac{1}{\sqrt{10}}, \frac{1}{\sqrt{10}}, \frac{1}{\sqrt{10}}\right)(0, v)\\
&= \begin{bmatrix}
\frac{2}{\sqrt{10}} & \frac{2}{\sqrt{10}} & \frac{-2}{\sqrt{10}} & \frac{-2}{\sqrt{10}}\\
\frac{2}{\sqrt{10}} & \frac{4}{\sqrt{10}} & \frac{6}{\sqrt{10}} & \frac{8}{\sqrt{10}}
\end{bmatrix} \cdot \begin{bmatrix}
0\\
0\\
v_1\\
v_2
\end{bmatrix}\\
&= \begin{bmatrix}
\frac{-2v_1}{\sqrt{10}} - \frac{2v_2}{\sqrt{10}}\\
\frac{6v_1}{\sqrt{10}} + \frac{8v_2}{\sqrt{10}}
\end{bmatrix}
\end{align*}

Now suppose there exists $v, w \in \mathbb{R}^2$ such that $L_2(v) = L_2(w)$. Then we have,
\begin{align*}
&\begin{bmatrix}
\frac{-2v_1}{\sqrt{10}} - \frac{2v_2}{\sqrt{10}}\\
\frac{6v_1}{\sqrt{10}} + \frac{8v_2}{\sqrt{10}}
\end{bmatrix} = \begin{bmatrix}
\frac{-2w_1}{\sqrt{10}} - \frac{2w_2}{\sqrt{10}}\\
\frac{6w_1}{\sqrt{10}} + \frac{8w_2}{\sqrt{10}}
\end{bmatrix}\\
\iff &\begin{bmatrix}
-2v_1 - 2v_2\\
6v_1 + 8v_2
\end{bmatrix} = \begin{bmatrix}
-2w_1 - 2w_2\\
6w_1 + 8w_2
\end{bmatrix}
\end{align*}

Solving the above system of equations yields $v_1 = w_1$ and $v_2 = w_2$. Hence $v = w$ and $L_2(v)$ is injective.\\

Now let $w \in \mathbb{R}^2$. Consider the equation,
\begin{align*}
L_2(v) = w
\end{align*}

for some $v \in \mathbb{R}^2$. This is equivalent to,
\begin{align*}
\begin{bmatrix}
\frac{-2v_1}{\sqrt{10}} - \frac{2v_2}{\sqrt{10}}\\
\frac{6v_1}{\sqrt{10}} + \frac{8v_2}{\sqrt{10}}
\end{bmatrix} &= \begin{bmatrix}
w_1 \\ w_2
\end{bmatrix}
\end{align*}

The first equation yields $v_1 = -\frac{\sqrt{10}}{2}w_1 - v_2$. Plugging into the second equation gives,
\begin{align*}
v_2 = \sqrt{\frac{5}{2}} (3w_1 + w_2)
\end{align*}

Plugging this back into the previous equation yields,
\begin{align*}
v_1 &= -\frac{\sqrt{10}}{2} w_1 - \sqrt{\frac{5}{2}} (3w_1 + w_2)\\
&= \frac{1}{2} \sqrt{10} (-4w_1 - w_2)
\end{align*}

We have that $w_1, w_2 \in \mathbb{R}$ and $\mathbb{R}$ is closed under addition and multiplication. Hence, $v_1, v_2 \in \mathbb{R}$ and so $v \in \mathbb{R}^2$.\\

Thus, for every $w \in \mathbb{R}^2$, $\exists v \in \mathbb{R}^2$ such that $L_2(v) = w$. As a result, $L_2$ is a bijection from $\mathbb{R}^2$ to $\mathbb{R}^2$.\\

Hence, we can apply the Implicit Function Theorem here. As a result, there exists a neighborhood $W$ of $\left(\frac{1}{\sqrt{10}}, \frac{1}{\sqrt{10}}\right) \in \mathbb{R}^2$ and a unique function $\varphi: W \to \mathbb{R}^2$ belonging to class $C^1(W)$ such that $(u, v) = \varphi(x, y)$ and,
\begin{align*}
F(x, y, \varphi_1(x, y), \varphi_2(x, y)) = 0 \; \; \forall v \in \mathbb{R}^2
\end{align*}

where $u(x, y) = \varphi_1(x,y)$ and $v(x, y) = \varphi_2(x, y)$.\\

We will now use Corollary 41.10 to calculate $Du(x, y) = D\varphi_1(x, y)$ and $Dv(x, y) = \varphi_2(x, y)$,
\begin{align*}
D\varphi(x, y) &= -[D_{(2)} F(x, y, \varphi_1(x,y), \varphi_2(x,y)]^{-1} \circ [D_{(1)} F(x, y, \varphi_1(x,y), \varphi_2(x,y)]\\
&= -\frac{1}{-4uv} \begin{bmatrix}
8v & 2v\\
-6u & -2u
\end{bmatrix} \circ \begin{bmatrix}
2x & 2y\\
2x & 4y
\end{bmatrix}\\
&= -\frac{1}{-4uv} \begin{bmatrix}
16xv + 4xv & 16vy + 32vy\\
-12ux - 4ux & -12uy - 8uy
\end{bmatrix}\\
&= -\frac{1}{-4uv} \begin{bmatrix}
20xv & 48vy\\
-16ux & -20uy
\end{bmatrix}\\
&= \begin{bmatrix}
\frac{-5x}{u} & \frac{-12y}{u}\\
\frac{4x}{v} & \frac{5y}{v}
\end{bmatrix}
\end{align*}

Hence, 
\begin{align*}
Du(x,y) = D\varphi_1 = \begin{bmatrix}
\frac{-5x}{u} & \frac{-12y}{u}
\end{bmatrix}
\end{align*}

and,
\begin{align*}
Dv(x,y) = D\varphi_2 = \begin{bmatrix}
\frac{4x}{v} & \frac{5y}{v}
\end{bmatrix}
\end{align*}

\begin{problem}{5}
\end{problem}

We have $L \in \mathscr{L}(\mathbb{R}^n, \mathbb{R}^n)$ and $r > 0$ such that $||L(x)|| \geq 2r ||x|| \; \; \forall x \in \mathbb{R}^n$.\\

We need to show that $\exists \epsilon >0$ such that if $L' \in \mathscr{L}(\mathbb{R}^n, \mathbb{R}^n)$ with $|||L' - L||| < \epsilon$ (note that we are using the operator norm here), then
\begin{align}
||L'(x)|| \geq r||x|| \; \; \forall x \in \mathbb{R}^n
\end{align}

Define $\epsilon = r$ and let $L' \in \mathscr{L}(\mathbb{R}^n, \mathbb{R}^n)$ such that $|||L' - L||| < \epsilon$.\\

For the first case, we consider $x = 0$. We have that (1) clearly holds.\\

For the second case, consider $||x|| = 1$. Then we have,
\begin{align*}
||L'(x)|| &= ||L(x) - L(x) + L'(x)||\\
&\geq ||L(x)|| - ||L(x) - L'(x)||\\
&\geq 2r||x|| - r\\
&= r||x||
\end{align*}

Note that the last line only holds because we assumed $||x|| = 1$.\\

Now consider the case $x \in \mathbb{R}^n \setminus \{0\}$ with $||x|| \neq 1$.\\

Define $y = \frac{x}{||x||}$, and so $||y|| = 1$. Hence, from case 2, we have,
\begin{align}
||L'(y)|| \geq r||y|| = \frac{1}{||x||} r||x||
\end{align}

In addition, by properties of linear transformations, we have,
\begin{align}
||L'(y)|| = \left|\left| \frac{1}{||x||} L'(x)\right|\right| = \frac{1}{||x||} ||L'(x)||
\end{align}

Combining (2) and (3) yields,
\begin{align*}
&\frac{1}{||x||} ||L'(x)|| \geq \frac{1}{||x||} r||x||\\
\iff &||L'(x)|| \geq r||x||
\end{align*}

as required.\\

We have proven that the required property holds when $||x|| = 0, ||x|| = 1$, and $||x|| \neq 1, 0$. Hence, we have proven this statement for all possible cases of $x \in \mathbb{R}^n$.

\end{document}