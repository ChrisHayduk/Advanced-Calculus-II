\documentclass[12pt]{article}
 
\usepackage[margin=1in]{geometry}
\usepackage{amsmath,amsthm,amssymb, mathtools}
\usepackage[T1]{fontenc}
\usepackage{lmodern}
\usepackage{fixltx2e}
\usepackage[shortlabels]{enumitem}
\usepackage{mathrsfs}
 
\newcommand{\N}{\mathbb{N}}
\newcommand{\R}{\mathbb{R}}
\newcommand{\Z}{\mathbb{Z}}
\newcommand{\Q}{\mathbb{Q}}
 
\newenvironment{theorem}[2][Theorem]{\begin{trivlist}
\item[\hskip \labelsep {\bfseries #1}\hskip \labelsep {\bfseries #2.}]}{\end{trivlist}}
\newenvironment{lemma}[2][Lemma]{\begin{trivlist}
\item[\hskip \labelsep {\bfseries #1}\hskip \labelsep {\bfseries #2.}]}{\end{trivlist}}
\newenvironment{exercise}[2][Exercise]{\begin{trivlist}
\item[\hskip \labelsep {\bfseries #1}\hskip \labelsep {\bfseries #2.}]}{\end{trivlist}}
\newenvironment{problem}[2][Problem]{\begin{trivlist}
\item[\hskip \labelsep {\bfseries #1}\hskip \labelsep {\bfseries #2.}]}{\end{trivlist}}
\newenvironment{question}[2][Question]{\begin{trivlist}
\item[\hskip \labelsep {\bfseries #1}\hskip \labelsep {\bfseries #2.}]}{\end{trivlist}}
\newenvironment{corollary}[2][Corollary]{\begin{trivlist}
\item[\hskip \labelsep {\bfseries #1}\hskip \labelsep {\bfseries #2.}]}{\end{trivlist}}
\newcommand{\textfrac}[2]{\dfrac{\text{#1}}{\text{#2}}}

\begin{document}

\title{Advanced Calculus II: Assignment 4}

\author{Chris Hayduk}
\date{\today}

\maketitle

\begin{problem}{1}
\end{problem}

Suppose $g: \mathbb{R} \to \mathbb{R}$ is differentiable with $g'(x) \neq 0$ for all $x \in \mathbb{R}$.\\

Now let $x_1, x_2 \in \mathbb{R}$ such that $x_1 \neq x_2$ and $g(x_1) = g(x_2)$. Assume without loss of generality that $x_1 < x_2$.\\

Take $h(x) = g(x) - g(x_1)$. We have that $h(x)$ is clearly defined and differentiable for all $x \in \mathbb{R}$ because it is a linear combination of a constant and a differentiable function.\\

In addition, note that $h(x_1) = 0 = h(x_2)$. Hence, we can apply Rolle's Theorem on the interval $[x_1, x_2]$. As a result, we have that $\exists c \in [x_1, x_2]$ such that $h'(c) = 0$.\\

Now observe the defintion of the dervative of $h$ at $c$:
\begin{align*}
h'(c) &= \lim_{x \to c} \frac{h(x) - h(c)}{x - c}\\
&= \lim_{x \to c} \frac{[g(x) - g(x_1)] - [g(c) - g(x_1)]}{x - c}\\
&= \lim_{x \to c} \frac{g(x) - g(c)}{x - c}\\
&= g'(c)
\end{align*}

Hence, we have that $h'(c) = 0 = g'(c)$. However, we know that $g'(x) \neq 0$ for all $x \in \mathbb{R}$. Thus, we have a contradiction and we must have that $x_1 = x_2$. That is, $g(x)$ is injective from $\mathbb{R} \to g(\mathbb{R})$.\\

By the definition of $g(\mathbb{R})$ (ie. the range of $g$), we have that $g$ is surjective from $\mathbb{R} \to g(\mathbb{R})$.\\

As a result, $g$ is a bijection of $\mathbb{R}$ onto $g(\mathbb{R})$.

\begin{problem}{2}
\end{problem}

Let $g(x) = f(x) - C(x - a)$. We have that $g(x)$ is a linear combination of continuous functions on $[a, b]$, so it is continuous on $[a, b]$ as well.\\

Thus, since $g$ is continuous on the compact set $[a, b]$, it attains its maximum value on the interval $[a, b]$ by the extreme value theorem.\\

We have $g'(x) = f'(x) - C$. Note that $g'(a) = f'(a) - C$. Since $f'(a) < C < f'(b)$, we have $g'(a) = f'(a) - C < 0$. By the Interior Maximum Theorem, then $g(a)$ cannot be the maximum value of $g$.\\

Similarly, we have that $g'(b) = f'(b) - C(b-a) > 0$, hence $g(b)$ cannot be the maximum value of $g$.\\

Thus, $g$ must attain its maximum value at some point in $c \in (a, b)$. At this point, we have that $g'(c) = 0$, again by Interior Maximum Theorem, which yields,
\begin{align*}
&g'(c) = 0\\
\implies &f'(c) - C = 0\\
\implies &f'(c) = C
\end{align*}

\begin{problem}{3}
\end{problem}

Show that $f$ is partial differentiable at $(0, 0)$ with respect to any $(a, b) \in \mathbb{R}^2$:\\

Let $u = (a, b)$ be arbitrary in $\mathbb{R}^2$ with $(a, b) \neq (0, 0)$. In addition, let $c = (0, 0)$. Then, by Definition 39.1 in the text, we have
\begin{align*}
L_u &= \lim_{t \to 0} \frac{1}{t} [f(c + tu) - f(c)]\\
&= \lim_{t \to 0} \frac{1}{t} [f(ta, tb) - f(0, 0)]\\
&= \lim_{t \to 0} \frac{1}{t} \left[\frac{(ta)(tb)^2}{(ta)^2 + (tb)^4}\right]\\
&= \lim_{t \to 0} \frac{1}{t} \left[\frac{t^3ab^2}{t^2a^2 + t^4b^4}\right]\\
&= \lim_{t \to 0} \frac{t^3ab^2}{t^3a^2 + t^5b^4}\\
&= \lim_{t \to 0} \frac{ab^2}{a^2 + t^2b^4}\\
&= \frac{ab^2}{a^2}\\
&= \frac{b^2}{a}
\end{align*}

If $u = (0, 0)$, then clearly we have,
\begin{align*}
L_u &= \lim_{t \to 0} \frac{1}{t} [f(c + tu) - f(c)]\\
&= \lim_{t \to 0} \frac{1}{t} [f(0, 0) - f(0, 0)]\\
&= \lim_{t \to 0} \frac{1}{t} (0)\\
&= 0
\end{align*}

Hence, $L_u$ is defined at $(0, 0)$ for every $u \in \mathbb{R}^2$.\\

Show that $f$ is not continuous at $(0,0)$:\\

Observe that $f$ is defined for every point in $\mathbb{R}^2$ except for $(0, 0)$. Hence, we can approach $(0, 0)$ on any line. Thus, let $x = y^2$. Then, we have,
\begin{align*}
f(y^2, y) &= \frac{y^2y^2}{(y^2)^2 + y^4}\\
&= \frac{y^4}{y^4 + y^4}\\
&= \frac{y^4}{y^4(1 + 1)}\\
&= \frac{1}{2}
\end{align*}

Thus, $\lim_{(y^2, y) \to (0, 0)} f(x, y) = \frac{1}{2}$.\\

Now approach on the line $x = 0$. This yields,
\begin{align*}
f(0, y) &= \frac{(0)y^2}{(0)^2 + y^4}\\
&= \frac{0}{y^4}\\
&= 0
\end{align*}

Hence, $\lim_{(0, y) \to (0, 0)} f(x, y) = 0 \neq \lim_{(y^2, y) \to (0, 0)} f(x, y)$.\\

We can see that the limit as we approach $(0, 0)$ is different depending upon the line we approach it from. Hence $f$ is not continuous at $(0, 0)$.
\newpage
\begin{problem}{4}
\end{problem}

By Corollary 39.7 in the text, since $f$ is differentiable at $c$, we have that $Df(c)(u) = u_1D_1f(c) + \cdots + u_nD_nf(c)$.\\

Thus, let $v_c = \begin{bmatrix}D_1f(c) \\ \vdots \\ D_nf(c) \end{bmatrix}$. Then clearly we have,
\begin{align*}
Df(c)(u) = v_c \cdot u
\end{align*}

where $u = \begin{bmatrix} u_1, & \cdots, & u_n \end{bmatrix} \in \mathbb{R}^n$. Now need to show that $v_c$ is unique.\\

Suppose there exists $v_c' \neq v_c$ such that
\begin{align*}
Df(c)(u) = v_c' \cdot u
\end{align*} 

Then we have that,
\begin{align*}
v_c \cdot u = v_c' \cdot u
\end{align*}

That is,
\begin{align*}
D_1f(c) u_1 &= v_1' u_1\\
&\vdots\\
D_nf(c) u_n &= v_n' u_n
\end{align*}

Each equation above is an equation of real numbers, so by using the field properties we can divide both sides by $u_i$ for each $i \in \{1, \cdots, n\}$, which yields,
\begin{align*}
D_1f(c) &= v_1' \\
&\vdots\\
D_nf(c) &= v_n' 
\end{align*}

Hence, we have that $v_c = v_c'$ and thus the gradient of $f$ at $c$ is unique.

\newpage
\begin{problem}{5}
\end{problem}

Let $Df_1$ and $Df_2$ exist at $c$ and let $Df_1$ be continuous in a neighborhood around $c$.
\end{document}