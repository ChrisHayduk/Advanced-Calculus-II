\documentclass[12pt]{article}
 
\usepackage[margin=1in]{geometry}
\usepackage{amsmath,amsthm,amssymb, mathtools}
\usepackage[T1]{fontenc}
\usepackage{lmodern}
\usepackage{fixltx2e}
\usepackage[shortlabels]{enumitem}
\usepackage{mathrsfs}
 
\newcommand{\N}{\mathbb{N}}
\newcommand{\R}{\mathbb{R}}
\newcommand{\Z}{\mathbb{Z}}
\newcommand{\Q}{\mathbb{Q}}
 
\newenvironment{theorem}[2][Theorem]{\begin{trivlist}
\item[\hskip \labelsep {\bfseries #1}\hskip \labelsep {\bfseries #2.}]}{\end{trivlist}}
\newenvironment{lemma}[2][Lemma]{\begin{trivlist}
\item[\hskip \labelsep {\bfseries #1}\hskip \labelsep {\bfseries #2.}]}{\end{trivlist}}
\newenvironment{exercise}[2][Exercise]{\begin{trivlist}
\item[\hskip \labelsep {\bfseries #1}\hskip \labelsep {\bfseries #2.}]}{\end{trivlist}}
\newenvironment{problem}[2][Problem]{\begin{trivlist}
\item[\hskip \labelsep {\bfseries #1}\hskip \labelsep {\bfseries #2.}]}{\end{trivlist}}
\newenvironment{question}[2][Question]{\begin{trivlist}
\item[\hskip \labelsep {\bfseries #1}\hskip \labelsep {\bfseries #2.}]}{\end{trivlist}}
\newenvironment{corollary}[2][Corollary]{\begin{trivlist}
\item[\hskip \labelsep {\bfseries #1}\hskip \labelsep {\bfseries #2.}]}{\end{trivlist}}
\newcommand{\textfrac}[2]{\dfrac{\text{#1}}{\text{#2}}}

\begin{document}

\title{Advanced Calculus II: Assignment 3}

\author{Chris Hayduk}
\date{\today}

\maketitle

\begin{problem}{1}
\end{problem}

Let $L_1, L_2, L_3 \in L(\mathbb{R}^n, \mathbb{R}^m)$. Then all of these functions are linear transformation from $\mathbb{R}^n$ to $\mathbb{R}^m$. We will now use these properties to show that $L(\mathbb{R}^n, \mathbb{R}^m)$ is a vector space:

\begin{enumerate}

\item Associativity of addition
\begin{align*}
(L_1 + L_2)(x) + L_3(x) &= L_1(x) + L_2(x) + L_3(x)\\
&= L_1(x) + (L_2 + L_3)(x)
\end{align*}

\item Commutativity of addition

\item Identity element of addition\\

Let $L_0$ be the function that assigns the $0$ vector in $\mathbb{R}^m$ to every vector in $\mathbb{R}^n$. We must first show that this is a linear transformation.\\

Let $u, v \in \mathbb{R}^n$ and let $c \in \mathbb{R}$. Then,
\begin{align*}
L_0(u + v) = 0 = L_0(u) + L_0(v)
\end{align*}

and,
\begin{align*}
L_0(cu) = 0 = c0 = cL(u)
\end{align*}

Thus $L_0 \in L(\mathbb{R}^n, \mathbb{R}^m)$. Now to show that it is the identity element of addition in that set:
\begin{align*}
(L_0 + L_1)(x) &= L_0(x) + L_1(x)\\
&= 0 + L_1(x)\\
&= L_1(x)
\end{align*}

\item Inverse elements of addition

\end{enumerate}

\begin{problem}{2}
\end{problem}


\begin{problem}{3}
\end{problem}


\begin{problem}{4}
\end{problem}


\begin{problem}{5}
\end{problem}


\end{document}