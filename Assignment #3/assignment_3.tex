\documentclass[12pt]{article}
 
\usepackage[margin=1in]{geometry}
\usepackage{amsmath,amsthm,amssymb, mathtools}
\usepackage[T1]{fontenc}
\usepackage{lmodern}
\usepackage{fixltx2e}
\usepackage[shortlabels]{enumitem}
\usepackage{mathrsfs}
 
\newcommand{\N}{\mathbb{N}}
\newcommand{\R}{\mathbb{R}}
\newcommand{\Z}{\mathbb{Z}}
\newcommand{\Q}{\mathbb{Q}}
 
\newenvironment{theorem}[2][Theorem]{\begin{trivlist}
\item[\hskip \labelsep {\bfseries #1}\hskip \labelsep {\bfseries #2.}]}{\end{trivlist}}
\newenvironment{lemma}[2][Lemma]{\begin{trivlist}
\item[\hskip \labelsep {\bfseries #1}\hskip \labelsep {\bfseries #2.}]}{\end{trivlist}}
\newenvironment{exercise}[2][Exercise]{\begin{trivlist}
\item[\hskip \labelsep {\bfseries #1}\hskip \labelsep {\bfseries #2.}]}{\end{trivlist}}
\newenvironment{problem}[2][Problem]{\begin{trivlist}
\item[\hskip \labelsep {\bfseries #1}\hskip \labelsep {\bfseries #2.}]}{\end{trivlist}}
\newenvironment{question}[2][Question]{\begin{trivlist}
\item[\hskip \labelsep {\bfseries #1}\hskip \labelsep {\bfseries #2.}]}{\end{trivlist}}
\newenvironment{corollary}[2][Corollary]{\begin{trivlist}
\item[\hskip \labelsep {\bfseries #1}\hskip \labelsep {\bfseries #2.}]}{\end{trivlist}}
\newcommand{\textfrac}[2]{\dfrac{\text{#1}}{\text{#2}}}

\begin{document}

\title{Advanced Calculus II: Assignment 3}

\author{Chris Hayduk}
\date{\today}

\maketitle

\begin{problem}{1}
\end{problem}

Let $L_1, L_2, L_3 \in L(\mathbb{R}^n, \mathbb{R}^m)$. Then all of these functions are linear transformation from $\mathbb{R}^n$ to $\mathbb{R}^m$. We will now use these properties to show that $L(\mathbb{R}^n, \mathbb{R}^m)$ is a vector space:

\begin{enumerate}

\item Associativity of addition
\begin{align*}
(L_1 + L_2)(x) + L_3(x) &= L_1(x) + L_2(x) + L_3(x)\\
&= L_1(x) + (L_2 + L_3)(x)
\end{align*}

\item Commutativity of addition

\begin{align*}
(L_1 + L_2)(x) &= L_1(x) + L_2(x)
\end{align*}

We have that $L_1(x), L_2(x) \in \mathbb{R}^m$ for every $x \in \mathbb{R}^n$. Since $\mathbb{R}^m$ is a vector space, we must have that,
\begin{align*}
(L_1 + L_2)(x) &= L_1(x) + L_2(x)\\
&= L_2(x) + L_1(x)\\
&= (L_2 + L_1)(x)
\end{align*}

as required.

\item Identity element of addition

Let $L_0$ be the function that assigns the $0$ vector in $\mathbb{R}^m$ to every vector in $\mathbb{R}^n$. We must first show that this is a linear transformation.\\

Let $u, v \in \mathbb{R}^n$ and let $c \in \mathbb{R}$. Then,
\begin{align*}
L_0(u + v) = 0 = L_0(u) + L_0(v)
\end{align*}

and,
\begin{align*}
L_0(cu) = 0 = c0 = cL(u)
\end{align*}

Thus $L_0 \in L(\mathbb{R}^n, \mathbb{R}^m)$. Now to show that it is the identity element of addition in that set:
\begin{align*}
(L_0 + L_1)(x) &= L_0(x) + L_1(x)\\
&= 0 + L_1(x)\\
&= L_1(x)
\end{align*}

\item Inverse elements of addition

First we will show that $-L_1(x)$ is a linear transformation.\\

Let $u, v \in \mathbb{R}^n$ and let $c \in \mathbb{R}$. Then,
\begin{align*}
-L_1(u + v) &= -(L_1(u) + L_1(v))\\
&= -L_1(u) + -L_1(v)
\end{align*}

and,
\begin{align*}
c(-L_1(u)) = (-c)L_1(u) = -L_1(cu)
\end{align*}

Thus $-L_1 \in L(\mathbb{R}^n, \mathbb{R}^m)$. Now to show that it is the inverse element of addition in that set:
\begin{align*}
(L_1 + -L_1)(x) &= L_1(x) + -L_1(x)\\
&= 0
\end{align*}

\item Compatibility of scalar multiplication with field multiplication

Let $a, b \in \mathbb{R}$. Then, using properties of linear transformations, we have
\begin{align*}
a(bL_1(x)) &= a(L_1(bx))\\
&= L_1(a(bx))\\
&= L_1((ab)x)\\
&= (ab)L_1(x)
\end{align*}

\item Identity element of scalar multiplication	

Let $1 \in \mathbb{R}$. Since $\mathbb{R}^n$ is a vector space with $1$ as the scalar multiplication identity, we have that $1x = x$ for every $x \in \mathbb{R}^n$. Moreover, by properties of linear transformations, we have,

\begin{align*}
(1)L_1(x) &= L_1(1x)\\
&= L_1(x)
\end{align*}

\item Distributivity of scalar multiplication with respect to vector addition

Let $c \in \mathbb{R}$. Since $L_1(u), L_2(v) \in \mathbb{R}^m$ and $\mathbb{R}^m$ is an $\mathbb{R}$ vector space, we must necessarily have that 

\begin{align*}
c(L_1(x) + L_2(x)) = cL_1(x) + cL_2(x)
\end{align*}

\item Distributivity of scalar multiplication with respect to field addition

Let $a, b \in \mathbb{R}$. Moreover, let $d = a+b$. Then,
\begin{align*}
(a+ b)L_1(x) &= dL_1(x)\\
&= L_1(dx)\\
&= L_1((a+b)x)\\
&= L_1(ax + bx)\\
&= L_1(ax) + L_1(bx)\\
&= aL_1(x) + bL_1(x)
\end{align*}
\end{enumerate}

Hence, $L(\mathbb{R}^n, \mathbb{R}^m)$ together with the standard addition and scalar multiplication is an $\mathbb{R}$ vector space.

\begin{problem}{2}
\end{problem}

We will show that $||T|| = \sup \{||T(x)||: ||x|| \leq 1\}$ defines a norm on $L(\mathbb{R}^n, \mathbb{R}^m)$.\\

Let $T_1, T_2 \in L(\mathbb{R}^n, \mathbb{R}^m)$. We must check the three norm properties:

\begin{enumerate}

\item We have that,
\begin{align*}
||T_1 + T_2|| &= \sup \{||(T_1 + T_2)(x)||: ||x|| \leq 1 \}\\
&= \sup \{||T_1(x) + T_2(x)||: ||x|| \leq 1 \}
\end{align*}

Since $T_1(x), T_2(x) \in \mathbb{R}^m$ for every $x \in \mathbb{R}^n$, we have that $||T_1(x) + T_2(x)|| \leq ||T_1(x)|| + ||T_2(x)||$ for any valid norm on $\mathbb{R}^m$. Hence, we have,
\begin{align*}
||T_1 + T_2|| &= \sup \{||T_1(x) + T_2(x)||: ||x|| \leq 1 \}\\
&\leq \sup \{||T_1(x)|| + ||T_2(x)||: ||x|| \leq 1 \}\\
&= ||T_1|| + ||T_2||
\end{align*}

\item Let $a \in \mathbb{R}$. By properties of norms and supremum, we have

\begin{align*}
||aT_1|| &= \sup \{||aT_1(x)||: ||x|| \leq 1 \}\\
&= \sup \{|a| \cdot ||T_1(x)||: ||x|| \leq 1 \}\\
&= |a| \sup \{||T_1(x)||: ||x|| \leq 1 \}\\
&= |a| \cdot ||T_1||
\end{align*}

\item Suppose $||T|| = 0$. Then need to show that $T = \boldsymbol{0}$ where $\boldsymbol{0}$ is the $0$ vector in $\mathbb{R}^m$. We have,

\begin{align*}
||T|| &= \sup \{||T(x)||: ||x|| \leq 1\}
\end{align*}

Since any valid norm on $\mathbb{R}^m$ is non-negative and the supremum of $||T(x)||$ with $||x|| \leq 1$ is $0$, we must have that $||T(x)|| = 0$ for all $||x|| \leq 1$ and hence $T(x) = 0$ for all $||x|| \leq 1$.\\

Now suppose $\exists x_1$ with $||x_1|| > 1$ and $||T(x_1)|| \neq 0$ (that is, $T(x_1) = 0$).\\

Write $x_1$ as a linear combination of basis vectors. Hence, $x_1 = c_1e_1 + c_2e_2 + \cdots + c_ne_n$. Then we have,
\begin{align*}
||T(x_1)|| &= ||T(c_1e_1 + c_2e_2 + \cdots + c_ne_n)||\\
&= ||T(c_1e_1) + \cdots + T(c_ne_n)||
&= ||c_1T(e1) + \cdots + c_nT(e_n)||
&= ||c_1(0) + \cdots c_n(0)||\\
&= ||0|| = 0
\end{align*}

However, we supposed that $||T(x_1)|| \neq 0$, a contradiction. Thus, $T(x) = \boldsymbol{0}$.
\end{enumerate}

Thus, the operator norm defines a valid norm on $L(\mathbb{R}^n, \mathbb{R}^m)$.

\begin{problem}{3}
\end{problem}


\begin{problem}{4}
\end{problem}

Suppose $f$ is even and differentiable. Then,

\begin{align*}
f'(-x) &= \lim_{h \to 0} \frac{f(-x+h) - f(-x)}{h}\\
&= \lim_{h \to 0} \frac{f(x-h) - f(x)}{h}\\
&= - \lim_{h \to 0} \frac{f(x) - f(x-h)}{h}\\
&= -f'(x)
\end{align*}

Hence, $f'$ is odd.


\begin{problem}{5}
\end{problem}

We have,
\begin{align*}
f'(0) &= \lim_{h \to 0} \frac{f(0 + h) - f(0)}{h}\\
&= \lim_{h \to 0} \frac{h^2 \sin(1/h) - 0}{h}\\
&= \lim_{h \to 0} h \sin(1/h)
\end{align*}

We have that $|\sin(1/h)| \leq 1$ for every value of $h$. Hence,

\begin{align*}
|h \sin(1/h)| \leq |h|
\end{align*}

Thus,
\begin{align*}
&- \lim_{h \to 0} |h| \leq \lim_{h \to 0} h \sin(1/h) \leq \lim_{h \to 0} |h|\\
\implies &0 \leq \lim_{h \to 0} h \sin(1/h) \leq 0\\
\implies &\lim_{h \to 0} h = 0
\end{align*}

Hence, $f'(0)$ exists and equals $0$.\\

Now need to show that $f'$ is not continuous at $0$. We have that,

\begin{align*}
f'(x) = \begin{cases} 
      2x \sin(1/x) - \cos(1/x) & x \neq 0 \\
      0 & x = 0
   \end{cases}
\end{align*}

Observe that,
\begin{align*}
\lim_{x \to 0} f'(x) &= \lim_{x \to 0} 2x \sin(1/x) - \cos(1/x)\\
&= \lim_{x \to 0} 2x \sin(1/x) - \lim_{x \to 0} \cos(1/x)\\
&= - \lim_{x \to 0} \cos(1/x)
\end{align*}

This limit is undefined, and so $f'(0) \neq \lim_{x \to 0} f'(x)$. Hence, $f'(x)$ is not continuous at $0$.


\end{document}