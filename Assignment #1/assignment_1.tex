\documentclass[12pt]{article}
 
\usepackage[margin=1in]{geometry}
\usepackage{amsmath,amsthm,amssymb, mathtools}
\usepackage[T1]{fontenc}
\usepackage{lmodern}
\usepackage{fixltx2e}
\usepackage[shortlabels]{enumitem}
\usepackage{mathrsfs}
 
\newcommand{\N}{\mathbb{N}}
\newcommand{\R}{\mathbb{R}}
\newcommand{\Z}{\mathbb{Z}}
\newcommand{\Q}{\mathbb{Q}}

\newcommand{\stcomp}[1]{{#1}^\complement}
 
\newenvironment{theorem}[2][Theorem]{\begin{trivlist}
\item[\hskip \labelsep {\bfseries #1}\hskip \labelsep {\bfseries #2.}]}{\end{trivlist}}
\newenvironment{lemma}[2][Lemma]{\begin{trivlist}
\item[\hskip \labelsep {\bfseries #1}\hskip \labelsep {\bfseries #2.}]}{\end{trivlist}}
\newenvironment{exercise}[2][Exercise]{\begin{trivlist}
\item[\hskip \labelsep {\bfseries #1}\hskip \labelsep {\bfseries #2.}]}{\end{trivlist}}
\newenvironment{problem}[2][Problem]{\begin{trivlist}
\item[\hskip \labelsep {\bfseries #1}\hskip \labelsep {\bfseries #2.}]}{\end{trivlist}}
\newenvironment{question}[2][Question]{\begin{trivlist}
\item[\hskip \labelsep {\bfseries #1}\hskip \labelsep {\bfseries #2.}]}{\end{trivlist}}
\newenvironment{corollary}[2][Corollary]{\begin{trivlist}
\item[\hskip \labelsep {\bfseries #1}\hskip \labelsep {\bfseries #2.}]}{\end{trivlist}}
\newcommand{\textfrac}[2]{\dfrac{\text{#1}}{\text{#2}}}

\begin{document}

\title{Advanced Calculus II: Assignment 1\\Chapter 2 - A Taste of Topology}

\author{Chris Hayduk}
\date{\today}

\maketitle

\begin{problem}{12 on p. 126}
\end{problem}

\begin{enumerate}[label=(\alph*)]
	\item The limit of a sequence is unaffected by rearrangement when $f$ is a bijective function. Since $f$ is bijective, each term from $(p_n)$ must be included one and only one time in the new sequence $(q_k)$.\\
	
	We know that, if $(p_n) \to \ell$, this means that $\forall \epsilon > 0, \exists N \in \mathbb{N}$ such that $n \geq N \implies \mathrm{d}(p_n, \ell) < \epsilon$.\\
	
	Thus, for each choice of $\epsilon > 0$ there are only finitely many terms for which $\mathrm{d}(p_n, \ell) \geq \epsilon$, while there are infinitely many terms for which $\mathrm{d}(p_n, \ell) < \epsilon$.\\
	
	As a result, the rearrangement $(q_k)$ will eventually exhaust all of the terms that have a distance from $\ell$ that is greater than or equal to $\epsilon$, and thus will have infinitely many terms left for which $\mathrm{d}(q_k, \ell) < \epsilon$.
	
	Hence, $(q_k) \to \ell$ as well.
	
	\item A rearrangement of $(p_n)$ where $f$ is an injective function does not necessarily preserve the limit of $(p_n)$. For example, take $(p_n) = (-1)^n$. This sequence alternates between 1 and -1, and thus never converges.\\
	
	Now let $f(n) = 2n$. This injective function maps the natural numbers to the evens. We can see that $\forall m$ where $m$ is even, we have that $p_m = 1$.\\
	
	Thus, $q_k = p_{f(k)} = 1 \; \forall k \in \mathbb{N}$. Clearly, $(q_k) \to 1$ while $(p_n)$ does not converge.
	
	\item A rearrangement of $(p_n)$ where $f$ is a surjective function does not necessarily preserve the limit of $(p_n)$. For example, take $(p_n) = \frac{1}{n}$. This sequence converges to 0.\\
	
	Now let $f(n) = \begin{cases} 
      1 & \mathrm{n \; is \; odd}\\
      2 & \mathrm{n = 2}\\
      f(n-2)+1 & \mathrm{n \; is \; even \; and \; n > 2}
   \end{cases}$\\
   
   Call this new sequence $q_k = p_{f(k)}$. Then we have that $(q_k)$ contains all terms in the original sequence $(p_n)$ and, for every odd term $m$, $q_m = 1$. Thus, the even terms of $(q_k)$ converge to 0 while the odd terms converge to 1. Since we have two subsequences in $(q_k)$ that converge to different limits, $(q_k)$ does not converge.
\end{enumerate}

\begin{problem}{44 on p. 128}
\end{problem}

\begin{enumerate}[label=(\alph*)]
	\item Since $f$ is continuous, then $\forall (p_n) \in M$ with a limit $p \in M$, we have $(p_n) \to p \implies f(p_n) \to f(p)$.\\
	
	Thus, for the graph of $f$, we have that $(p_n, f(p_n)) \to (p, f(p)) \in M \times f(M)$ for every sequence $(p_n, f(p_n)) \in M \times \mathbb{R}$.\\
	
	Hence, the continuity of $f$ implies that its graph is closed.
	
	\item Suppose $M$ is compact and $f$ is continuous. By Theorem 38, the continuous image of a compact set is compact. Thus, $f(M)$ is a compact subset of $\mathbb{R}$.\\
	
	Hence, the graph of $f$ is the Cartesian product of two compact sets, $M$ and $f(M)$. By Corollary 29, the graph of $f$ is compact as a result.
	
	\item Assume the graph of $f$ is compact. Since the graph is compact, the graph must be bounded. By Bolzano-Weierstrass, any sequence in the graph must have a convergent subsequence. Since compact also implies closed, the limit of this subsequence must reside in the graph.\\
	
	Let $(p_n, f(p_n))$ be one such sequence in the graph. Thus, $\exists (p_{n_k})$ such that $(p_{n_k}, f(p_{n_k}))$ converges to a limit point.\\
	
	The limit point must be a limit point in each coordinate, so we have $(p_{n_k}, f(p_{n_k})) \to (p, f(p))$. Thus, $f$ preserves sequential limits and, as a result, $f$ is continuous.
	
	\item Counterexample:\\
	$f: [0, 1] \to \mathbb{R}$\\
	
	$f(x) = \begin{cases} 
      \frac{1}{x} & x \neq 0 \\
      0 & x = 0
   \end{cases}$
   
   $f$ is discontinuous because if you have a sequence $(x_n) \to 0$, $f(x_n) \to 
   \infty$ while $f(0) = 0$. Thus, $f(x_n) \not\to f(0)$. Hence, $f$ does not preserve sequential limits.
	
\end{enumerate}
\newpage

\begin{problem}{76 on p. 131}
\end{problem}

\begin{enumerate}[label=(\alph*)]
	\item Let $A = \{(x, y): x^2 + y^2 = 1\}$, i.e. the unit circle.\\
	
	Let $B = \{(x, 0): x \in \mathbb{R}\}$, i.e. a straight line along the x-axis.\\
	
	Both sets are connected. Their intersection is $A \cap B = \{(-1, 0), (1, 0)\}$. This set is disconnected in $\mathbb{R}^2$.
	
	\item Let $S_n = \{x: x \geq n\}$. Then each $S_n$ is connected and closed, and we have $S_1 \supset S_2 \supset \cdots$.\\
	
	However, we have $\cap S_n = \emptyset$.
	
	\item Intersection is connected. Prove this.\\
	
	Assume that $S_1, S_2, ...$ are a sequence of connected, compact sets with $S_1 \supset S_2 \supset \cdots$.\\
	
	By Theorem 34, their intersection $\cap S_n$ is compact and non-empty.
	
	\item Connected but not path connected. Prove this.
	
\end{enumerate}

\begin{problem}{1 on p. 147}
\end{problem}

We know that the intersection of a nested sequence of nonempty compact sets is a nonempty compact set. Thus, $\cap K_n$ is nonempty and compact. Hence, for every sequence $(a_n) \in \cap K_n$, there exists a subsequence $(a_{n_k})$ that converges to some limit $a$ in $\cap K_n$.\\

By properties of intersections, every term in $(a_{n_k})$ is in $K_n$ for every $n$. Furthermore, $a \in K_n$ for every $n$.\\

Now, by property (i), $f(K_n)$ is compact as well for every $n$. Thus, $\cap f(K_n)$ is compact.\\

We know that each term in $(a_{n_k})$ is in $K_n$ for every $n$. As a result, each $f(a_{n_k})$ is in $f(K_n)$ for every $n$. Thus, the image of the sequence $(a_{n_k})$ is contained in $\cap f(K_n)$. In addition, $f(a) \in \cap f(K_n)$.\\

Since $\cap f(K_n)$ is compact,

\end{document}