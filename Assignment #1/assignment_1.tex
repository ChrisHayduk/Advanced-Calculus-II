\documentclass[12pt]{article}
 
\usepackage[margin=1in]{geometry}
\usepackage{amsmath,amsthm,amssymb, mathtools}
\usepackage[T1]{fontenc}
\usepackage{lmodern}
\usepackage{fixltx2e}
\usepackage[shortlabels]{enumitem}
\usepackage{mathrsfs}
 
\newcommand{\N}{\mathbb{N}}
\newcommand{\R}{\mathbb{R}}
\newcommand{\Z}{\mathbb{Z}}
\newcommand{\Q}{\mathbb{Q}}
 
\newenvironment{theorem}[2][Theorem]{\begin{trivlist}
\item[\hskip \labelsep {\bfseries #1}\hskip \labelsep {\bfseries #2.}]}{\end{trivlist}}
\newenvironment{lemma}[2][Lemma]{\begin{trivlist}
\item[\hskip \labelsep {\bfseries #1}\hskip \labelsep {\bfseries #2.}]}{\end{trivlist}}
\newenvironment{exercise}[2][Exercise]{\begin{trivlist}
\item[\hskip \labelsep {\bfseries #1}\hskip \labelsep {\bfseries #2.}]}{\end{trivlist}}
\newenvironment{problem}[2][Problem]{\begin{trivlist}
\item[\hskip \labelsep {\bfseries #1}\hskip \labelsep {\bfseries #2.}]}{\end{trivlist}}
\newenvironment{question}[2][Question]{\begin{trivlist}
\item[\hskip \labelsep {\bfseries #1}\hskip \labelsep {\bfseries #2.}]}{\end{trivlist}}
\newenvironment{corollary}[2][Corollary]{\begin{trivlist}
\item[\hskip \labelsep {\bfseries #1}\hskip \labelsep {\bfseries #2.}]}{\end{trivlist}}
\newcommand{\textfrac}[2]{\dfrac{\text{#1}}{\text{#2}}}

\begin{document}

\title{Advanced Calculus II: Assignment 1\\Chapter 2 - A Taste of Topology}

\author{Chris Hayduk}
\date{\today}

\maketitle

\begin{problem}{12 on p. 126}
\end{problem}

\begin{enumerate}[label=(\alph*)]
	\item The limit of a sequence is unaffected by rearrangement when $f$ is a bijective function. Since $f$ is bijective, each term from $(p_n)$ must be included one and only one time in the new sequence $(q_k)$.\\
	
	We know that, if $(p_n) \to \ell$, this means that $\forall \epsilon > 0, \exists N \in \mathbb{N}$ such that $n \geq N \implies \mathrm{d}(p_n, \ell) < \epsilon$.\\
	
	Thus, for each choice of $\epsilon > 0$ there are only finitely many terms for which $\mathrm{d}(p_n, \ell) \geq \epsilon$, while there are infinitely many terms for which $\mathrm{d}(p_n, \ell) < \epsilon$.\\
	
	As a result, the rearrangement $(q_k)$ will eventually exhaust all of the terms that have a distance from $\ell$ that is greater than or equal to $\epsilon$, and thus will have infinitely many terms left for which $\mathrm{d}(q_k, \ell) < \epsilon$.
	
	Hence, $(q_k) \to \ell$ as well.
	
	\item A rearrangement of $(p_n)$ where $f$ is an injective function does not necessarily preserve the limit of $(p_n)$. For example, take $(p_n) = (-1)^n$. This sequence alternates between 1 and -1, and thus never converges.\\
	
	Now let $f(n) = 2n$. This injective function maps the natural numbers to the evens. We can see that $\forall m$ where $m$ is even, we have that $p_m = 1$.\\
	
	Thus, $q_k = p_{f(k)} = 1 \; \forall k \in \mathbb{N}$. Clearly, $(q_k) \to 1$ while $(p_n)$ does not converge.
	
	\item A rearrangement of $(p_n)$ where $f$ is a surjective function does not necessarily preserve the limit of $(p_n)$. For example, take $(p_n) = \frac{1}{n}$. This sequence converges to 0.\\
	
	Now let $f(n) = \begin{cases} 
      1 & \mathrm{n \; is \; odd}\\
      2 & \mathrm{n = 2}\\
      f(n-2)+1 & \mathrm{n \; is \; even \; and \; n > 2}
   \end{cases}$\\
   
   Call this new sequence $q_k = p_{f(k)}$. Then we have that $(q_k)$ contains all terms in the original sequence $(p_n)$ and, for every odd term $m$, $q_m = 1$. Thus, the even terms of $(q_k)$ converge to 0 while the odd terms converge to 1. Since we have two subsequences in $(q_k)$ that converge to different limits, $(q_k)$ does not converge.
\end{enumerate}

\begin{problem}{44 on p. 128}
\end{problem}



\begin{problem}{76 on p. 131}
\end{problem}

\begin{problem}{1 on p. 147}
\end{problem}




\end{document}