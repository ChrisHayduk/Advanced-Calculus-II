\documentclass[12pt]{article}
 
\usepackage[margin=1in]{geometry}
\usepackage{amsmath,amsthm,amssymb, mathtools}
\usepackage[T1]{fontenc}
\usepackage{lmodern}
\usepackage{fixltx2e}
\usepackage[shortlabels]{enumitem}
\usepackage{mathrsfs}
 
\newcommand{\N}{\mathbb{N}}
\newcommand{\R}{\mathbb{R}}
\newcommand{\Z}{\mathbb{Z}}
\newcommand{\Q}{\mathbb{Q}}
 
\newenvironment{theorem}[2][Theorem]{\begin{trivlist}
\item[\hskip \labelsep {\bfseries #1}\hskip \labelsep {\bfseries #2.}]}{\end{trivlist}}
\newenvironment{lemma}[2][Lemma]{\begin{trivlist}
\item[\hskip \labelsep {\bfseries #1}\hskip \labelsep {\bfseries #2.}]}{\end{trivlist}}
\newenvironment{exercise}[2][Exercise]{\begin{trivlist}
\item[\hskip \labelsep {\bfseries #1}\hskip \labelsep {\bfseries #2.}]}{\end{trivlist}}
\newenvironment{problem}[2][Problem]{\begin{trivlist}
\item[\hskip \labelsep {\bfseries #1}\hskip \labelsep {\bfseries #2.}]}{\end{trivlist}}
\newenvironment{question}[2][Question]{\begin{trivlist}
\item[\hskip \labelsep {\bfseries #1}\hskip \labelsep {\bfseries #2.}]}{\end{trivlist}}
\newenvironment{corollary}[2][Corollary]{\begin{trivlist}
\item[\hskip \labelsep {\bfseries #1}\hskip \labelsep {\bfseries #2.}]}{\end{trivlist}}
\newcommand{\textfrac}[2]{\dfrac{\text{#1}}{\text{#2}}}

\begin{document}

\title{Advanced Calculus II: Assignment 1\\Chapter 2 - A Taste of Topology}

\author{Chris Hayduk}
\date{\today}

\maketitle

\begin{problem}{12 on p. 126}
\end{problem}

\begin{enumerate}[label=(\alph*)]
	\item
	\item
	\item A rearrangement of $(p_n)$ where $f$ is a surjective function does not necessarily preserve the limit $\ell$. For example, take $(p_n) = \frac{1}{n}$. This sequence converges to 0.\\
	
	Now let $f(n) = \begin{cases} 
      1 & \mathrm{n \; is \; odd}\\
      2 & \mathrm{n = 2}\\
      f(n-2)+1 & \mathrm{n \; is \; even \; and \; n > 2}
   \end{cases}$\\
   
   Call this new sequence $q_k = p_{f(k)}$. Then we have that $(q_k)$ contains all terms in the original sequence $(p_n)$ and, for every odd term $m$, $q_m = 1$. Thus, the even terms of $(q_k)$ converge to 0 while the odd terms converge to 1. Since we have two subsequences in $(q_k)$ that converge to different limits, $(q_k)$ does not converge.
\end{enumerate}

\begin{problem}{44 on p. 128}
\end{problem}

\begin{problem}{76 on p. 131}
\end{problem}

\begin{problem}{1 on p. 147}
\end{problem}




\end{document}