\documentclass[12pt]{article}
 
\usepackage[margin=1in]{geometry}
\usepackage{amsmath,amsthm,amssymb, mathtools}
\usepackage[T1]{fontenc}
\usepackage{lmodern}
\usepackage{fixltx2e}
\usepackage[shortlabels]{enumitem}
\usepackage{mathrsfs}
 
\newcommand{\N}{\mathbb{N}}
\newcommand{\R}{\mathbb{R}}
\newcommand{\Z}{\mathbb{Z}}
\newcommand{\Q}{\mathbb{Q}}
 
\newenvironment{theorem}[2][Theorem]{\begin{trivlist}
\item[\hskip \labelsep {\bfseries #1}\hskip \labelsep {\bfseries #2.}]}{\end{trivlist}}
\newenvironment{lemma}[2][Lemma]{\begin{trivlist}
\item[\hskip \labelsep {\bfseries #1}\hskip \labelsep {\bfseries #2.}]}{\end{trivlist}}
\newenvironment{exercise}[2][Exercise]{\begin{trivlist}
\item[\hskip \labelsep {\bfseries #1}\hskip \labelsep {\bfseries #2.}]}{\end{trivlist}}
\newenvironment{problem}[2][Problem]{\begin{trivlist}
\item[\hskip \labelsep {\bfseries #1}\hskip \labelsep {\bfseries #2.}]}{\end{trivlist}}
\newenvironment{question}[2][Question]{\begin{trivlist}
\item[\hskip \labelsep {\bfseries #1}\hskip \labelsep {\bfseries #2.}]}{\end{trivlist}}
\newenvironment{corollary}[2][Corollary]{\begin{trivlist}
\item[\hskip \labelsep {\bfseries #1}\hskip \labelsep {\bfseries #2.}]}{\end{trivlist}}
\newcommand{\textfrac}[2]{\dfrac{\text{#1}}{\text{#2}}}

\begin{document}

\title{Advanced Calculus II: Assignment 1}

\author{Chris Hayduk}
\date{\today}

\maketitle

\begin{problem}{1}
\end{problem}

\begin{enumerate}[label=\alph*)]

\item 

\item Let $D_f \subset \mathbb{R}^n$ and let $f: D_f \to \mathbb{R}^m$. Also let $a \in D_f$.\\

Suppose $f$ is continuous at $a$ and suppose $|| \cdot ||_{n1}, || \cdot ||_{m1}$ are norms on $\mathbb{R}^n$ and $\mathbb{R}^m$ respectively.\\

Thus, by the definition of continuity, for every $\epsilon > 0$, there exists a $\delta > 0$ such that if $x \in D_f$ with $|| x - a ||_{n1} < \delta$, then $|| f(x) - f(a) ||_{m1} < \epsilon$.\\

By part $(a)$, for any other arbitrary norms $|| \cdot ||_{n2}, || \cdot ||_{m2}$ on $\mathbb{R}^n$ and $\mathbb{R}^m$, we have that
\begin{align*}
&C_{n1} || x - a ||_{n1} \leq || x - a ||_{n2} \leq C_{n2} || x - a ||_{n1}\\
&C_{m1} || f(x) - f(a) ||_{m1} \leq || f(x) - f(a) ||_{m2} \leq C_{m2} || f(x) - f(a) ||_{m1}
\end{align*}

for $C_{n1}, C_{n2}, C_{m1}, C_{m2} > 0$ and for every $x \in \mathbb{R}^n, f(x) \in \mathbb{R}^m$.\\

Hence, $|| x - a ||_{n1} < \delta \iff C_{n2} || x - a ||_{n1} < C_{n2} \delta \implies || x - a ||_{n2} < C_{n2} \delta$ for every $x \in \mathbb{R}^n$.\\

Now let $\delta' = \min\{\delta, C_{n2}\delta\}$.\\

Then clearly we have that $|| x - a ||_{n1} < \delta' \implies || f(x) - f(a) ||_{m1} < \epsilon$ and $ || x - a ||_{n2} < \delta' \implies || f(x) - f(a) ||_{m1} < \epsilon$.\\



\end{enumerate}

\begin{problem}{2}
\end{problem}

Suppose $A$ is open. Then for every $x \in A$, $\exists r > 0$ such that $B(x, r) \subset A$.\\

We know that a point $x' \in \mathbb{R}^n$ is a boundary point of $A$ if $\forall r > 0$, $\exists y_1, y_2 \in B(x', r)$ such that $y_1 \in A$ and $y_2 \in \mathbb{R}^n \setminus A$.\\

Now suppose there exists a point $z \in \partial A$ such that $z \in A$. Then, by definition of the openness of $A$, $\exists r_z > 0$ such that $B(z, r_z) \subset A$. However, since $z \in \partial A$, we also have that $\exists y_z \in \mathbb{R}^n \setminus A$ such that $y_z \in B(z, r_z)$. Hence, $B(z, r_z) \not\subset A$, a contradiction.\\

Thus we must have that, if $A$ is open, then $A \cap \partial A = \emptyset$.\\

Now suppose $A \cap \partial A = \emptyset$. Then $A$ does not contain any of its boundary points. That is, there is no $x \in \mathbb{R}^n$ such that $\forall r > 0, \exists y_1, y_2 \in B(x, r)$ such that $y_1 \in A$ and $y_2 \in \mathbb{R}^n \setminus A$. Note that there is always a $y_1 \in B(x, r)$ with $y_1 \in A$ since $x \in B(x, r)$ and $x \in A$.\\

Hence, for each $x \in A$, there must exist an $r > 0$ such that $B(x, r) \cap \left(\mathbb{R}^n \setminus A\right) = \emptyset$.\\

Thus, $B(x, r)$ must be contained in the complement of $\mathbb{R}^n \setminus A$, which is $A$. Hence, A is open.\\

Now suppose $A$ is closed. Then we have that $A^\mathsf{c} = \mathbb{R}^n \setminus A$ is open.\\

By the above proof, we have that $A^\mathsf{c} \cap \partial A^\mathsf{c} = \emptyset$. In addition, observe that the definition for $\partial A^\mathsf{c}$ is the same as for $\partial A$:\\

We know that a point $x \in \mathbb{R}^n$ is a boundary point of $A^\mathsf{c}$ if $\forall r > 0$, $\exists y_1, y_2 \in B(x, r)$ such that $y_1 \in A^\mathsf{c}$ and $y_2 \in \mathbb{R}^n \setminus A^\mathsf{c} = A$.\\

This precisely the same definition given for a boundary point of $A$. Hence, we can rewrite the statement from above as: $A^\mathsf{c} \cap \partial A^\mathsf{c} = A^\mathsf{c} \cap \partial A = \emptyset$.\\

Hence $\partial A \subset (A^\mathsf{c})^\mathsf{c} = A$, as required.\\

Now suppose $\partial A \subset A$. Take $A^\mathsf{c}$.\\

As we showed above, $\partial A = \partial A^\mathsf{c}$. Thus, we have $A^\mathsf{c} \cap \partial A = A^\mathsf{c} \cap \partial A^\mathsf{c} = \emptyset$.\\

By the earlier proof, we thus have that $A^\mathsf{c}$ is open. Then, by definition, $A$ is closed.

\begin{problem}{3}
\end{problem}

\begin{problem}{4}
\end{problem}

\begin{problem}{5}
\end{problem}

Let $f(x) = \begin{cases} 
      \frac{1}{q} & x = \frac{p}{q}, p, q \in \mathbb{Z} \\
      0 & x \text{ is irrational}
   \end{cases}$


\end{document}