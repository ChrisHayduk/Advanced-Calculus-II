\documentclass[12pt]{article}
 
\usepackage[margin=1in]{geometry}
\usepackage{amsmath,amsthm,amssymb, mathtools}
\usepackage[T1]{fontenc}
\usepackage{lmodern}
\usepackage{fixltx2e}
\usepackage[shortlabels]{enumitem}
\usepackage{mathrsfs}
 
\newcommand{\N}{\mathbb{N}}
\newcommand{\R}{\mathbb{R}}
\newcommand{\Z}{\mathbb{Z}}
\newcommand{\Q}{\mathbb{Q}}
 
\newenvironment{theorem}[2][Theorem]{\begin{trivlist}
\item[\hskip \labelsep {\bfseries #1}\hskip \labelsep {\bfseries #2.}]}{\end{trivlist}}
\newenvironment{lemma}[2][Lemma]{\begin{trivlist}
\item[\hskip \labelsep {\bfseries #1}\hskip \labelsep {\bfseries #2.}]}{\end{trivlist}}
\newenvironment{exercise}[2][Exercise]{\begin{trivlist}
\item[\hskip \labelsep {\bfseries #1}\hskip \labelsep {\bfseries #2.}]}{\end{trivlist}}
\newenvironment{problem}[2][Problem]{\begin{trivlist}
\item[\hskip \labelsep {\bfseries #1}\hskip \labelsep {\bfseries #2.}]}{\end{trivlist}}
\newenvironment{question}[2][Question]{\begin{trivlist}
\item[\hskip \labelsep {\bfseries #1}\hskip \labelsep {\bfseries #2.}]}{\end{trivlist}}
\newenvironment{corollary}[2][Corollary]{\begin{trivlist}
\item[\hskip \labelsep {\bfseries #1}\hskip \labelsep {\bfseries #2.}]}{\end{trivlist}}
\newcommand{\textfrac}[2]{\dfrac{\text{#1}}{\text{#2}}}

\begin{document}

\title{Advanced Calculus II: Assignment 5}

\author{Chris Hayduk}
\date{\today}

\maketitle

\begin{problem}{1}
\end{problem}

Suppose each $f$ is differentiable at $c \in D$. We know from Theorem 8.10 that for every $x \in \mathbb{R}^n$, we have that $|x_i| \leq ||x|| \leq \sqrt{n} \sup \{|x_1|, \cdots, |x_n|\}$.\\

Now let $\epsilon > 0$. By the differentiability of $f$ at $c$, $\exists \delta(\epsilon) > 0$ such that for every $u \in \mathbb{R}^n$ with $||u|| \leq \delta(\epsilon)$, we have,
\begin{align*}
\lim_{||u|| \to 0} \frac{||f(c+u) - f(c) - Df(c)(u)||}{||u||} = 0
\end{align*}

Note that $f(c+u)$, $f(c)$, and $Df(c)$ are all vectors in $\mathbb{R}^n$. Hence, Theorem 8.10 applies here and we have,
\begin{align}
\lim_{||u|| \to 0} \frac{|f_i(c+u) - f_i(c) - Df_i(c)(u)|}{||u||} \leq \lim_{||u|| \to 0} \frac{||f(c+u) - f(c) - Df(c)(u)||}{||u||} = 0
\end{align}

for every $i \in \{1, \cdots, n\}$.\\

Since $|f_i(c+u) - f_i(c) - Df_i(c)(u)|, \; ||u|| \geq 0$, we have from (1) that 

\begin{align*}
\lim_{||u|| \to 0} \frac{|f_i(c+u) - f_i(c) - Df_i(c)(u)|}{||u||} = 0
\end{align*}

Thus, each $f_i$ is differentiable at $c$.

Now assume that each $f_i$ is differentiable at $c$. We need to show that $f$ is differentiable at $c$.\\

By the second inequality in Theorem 8.10, we have that $||x|| \leq \sqrt{n} \sup \{|x_1|, \cdots, |x_n|\}$ for every $x \in \mathbb{R}^n$. \\

Let $|f_j(c+u) - f_i(c) - Df_i(c)(u)| = \sup \{|f_1(c+u) - f_1(c) - Df_1(c)(u)|, \cdots, |f_n(c+u) - f_n(c) - Df_n(c)(u)|\}$. Then,
\begin{align*}
\lim_{||u|| \to 0} \frac{||f(c+u) - f(c) - Df(c)(u)||}{||u||} &\leq \lim_{||u|| \to 0} \frac{\sqrt{n} \cdot |f_j(c+u) - f_j(c) - Df_{j}(c)(u)|}{||u||}\\
&= \sqrt{n} \lim_{||u|| \to 0} \frac{|f_j(c+u) - f_j(c) - Df_{j}(c)(u)|}{||u||}\\
&= \sqrt{n} (0) = 0
\end{align*}

The last line is obtained from the differentiablity of each $f_i$ at $c$.\\

Since $||f(c+u) - f(c) - Df(c)(u)||, \; ||u|| \geq 0$, we have from the above that,

\begin{align*}
\lim_{||u|| \to 0} \frac{||f(c+u) - f(c) - Df(c)(u)||}{||u||} = 0
\end{align*}

Hence, $f$ is differentiable at $c$.

\begin{problem}{2}
\end{problem}

Take the function $f(x, y) = \begin{cases} 
      (x^2 + y^2) \sin \left(\frac{1}{\sqrt{x^2 + y^2}} \right) & (x, y) \neq (0, 0) \\
      0 & (x, y) = (0, 0)
   \end{cases}$\\

$f$ is a composition of differentiable functions and hence is differentiable when $(x, y) \neq (0, 0)$. When $(x, y) = (0, 0)$, we have,
\begin{align*}
\lim_{||u|| \to 0} \frac{||f(0 + u_1, 0 + u_2) - f(0, 0) - L(u)||}{||u||} &= \lim_{||u|| \to 0} \frac{||(u_1^2 + u_2^2) \sin \left(\frac{1}{\sqrt{u_1^2 + u_2^2}} \right) - L(u)||}{||u||}\\
&\leq \lim_{||u|| \to 0} \frac{||(u_1^2 + u_2^2)||}{||u||} = 0
\end{align*}

Hence $f$ is also differentiable at $(0, 0)$.\\

Now consider the partial derivatives of $f$,
\begin{align*}
\frac{\partial f}{\partial x} (x, y) &= 2x \sin \left(\frac{1}{\sqrt{x^2 + y^2}} \right) - \frac{x \cos \left(\frac{1}{\sqrt{x^2 + y^2}} \right)}{\sqrt{x^2 + y^2}}\\
\frac{\partial f}{\partial y} (x, y) &= 2y \sin \left(\frac{1}{\sqrt{x^2 + y^2}} \right) - \frac{y \cos \left(\frac{1}{\sqrt{x^2 + y^2}} \right)}{\sqrt{x^2 + y^2}}
\end{align*}

Both of these functions oscillate wildly near the origin and are thus discontinuous.
\newpage
\begin{problem}{3}
\end{problem}

Suppose $h(x) = f(x) \cdot g(x)$ with $f, g: D \to \mathbb{R}^m$ differentiable at $c \in D$. Then we have,

\begin{align*}
&h(x) - h(c) - [Df(c)(x-c)g(c) + f(c)Dg(c)(x-c)]\\
&= [f(x) - f(c) - Df(c)(x-c)]g(x) + Df(c)(x-c)[g(x) - g(c)] +\\
 &f(c)[g(x) - g(c) - Dg(c)(x-c)]
\end{align*}

Since $g$ is continuous at $c$, we have that $\exists M$ such that $||g(x)|| < M$ for $||x - c|| < \delta$. Thus, if we choose $||x - c||$ small enough, we have that 
\begin{align*}
h(x) - h(c) - [Df(c)(x-c)g(c) + f(c)Dg(c)(x-c)] = 0
\end{align*}

and thus,
\begin{align*}
\lim_{x \to c} \frac{||h(x) - h(c) - [Df(c)(x-c)g(c) + f(c)Dg(c)(x-c)]||}{||x-c||} = 0
\end{align*}

Hence, $h$ is differentiable at $c$ with derivative
\begin{align*}
Dh(c)(x-c) = (Df(c)(x-c)) \cdot g(c) + f(c) \cdot (Dg(c)(x-c))
\end{align*}

\begin{problem}{4}
\end{problem}

Suppose $f$ differentiable on an open cell $J \subset \mathbb{R}^n$ and $f: J \to \mathbb{R}$. Also suppose that $D_1f(x) = 0$ for all $x \in J$. By Corollary 39.7, we have that, for every $c \in J$,

\begin{align*}
Df(c)(u) &= u_1D_1f(c) + u_2D_2f(c) + \cdots + u_nD_nf(c)\\
&= 0 + u_2D_2f(c) + \cdots + u_nD_nf(c)\\
&= u_2D_2f(c) + \cdots + u_nD_nf(c)
\end{align*}

Now let $y, z \in J$ with $y_{2} = z_{2}, \cdots, y_{n} = z_{n}$. Also let $\epsilon > 0$. Then $\exists \delta_1(\epsilon), \delta_2(\epsilon) > 0$ such that,
\begin{align}
||f(x) - f(y) - Df(y)(x - y)|| \leq \epsilon ||x - y||
\end{align}

and
\begin{align}
||f(x) - f(z) - Df(z)(x - z)|| \leq \epsilon ||x - z||
\end{align}

Now, using the facts that $D_1f(x) = 0$ for all $x \in J$ and that $y_{2} = z_{2}, \cdots, y_{n} = z_{n}$ along with Corollary 39.7, we have that,
\begin{align*}
Df(y)(x - y) &= (x_1 - y_1)D_1f(y) + (x_2 - y_2)D_2f(y) + \cdots + (x_n - y_n)D_nf(y)\\
&= (x_2 - y_2)D_2f(y) + \cdots + (x_n - y_n)D_nf(y)\\
&= (x_2 - z_2)D_2f(y) + \cdots + (x_n - z_n)D_nf(y)
\end{align*}
Using (2) and (3) along with the above equations yields,
\begin{align*}
&||f(x) - f(y) - Df(y)(x - y) - (f(x) - f(z) - Df(z)(x - z))||\\
&= ||f(z) - f(y) + Df(z)(x - y) - Df(y)(x - y)||\\
&= ||f(z) - f(y)||\\
&\leq ||f(x) - f(y) - Df(y)(x - y)|| + ||f(x) - f(z) - Df(z)(x - z)|| \\
&\leq \epsilon(||x - y|| + ||x - z||)
\end{align*}

Since $\epsilon(||x-y|| + ||x-z||)$ can be made arbitrarily small, we have $f(y) = f(z)$.

\end{document}